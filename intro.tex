\section{Introduction}
Game Theory is the study of mathematical models of conflict and cooperation between multiple agents who are considered as intelligent and rational decision-makers, and is used in a large range of domains (economics, psychology, computer science...). In communication networks, the protocols are seen as games in which the involved entities of the network must cooperatively follow the protocol rules. However, it may be possible that a particular entity is corrupted or does not work correctly with respect to the protocol. Some device may have been designed to deviate from the protocol to maximise a self-interest function or may be simply misconfigured. A protocol is unlikely to work correctly with respect to a property in the presence of selfish or ``broken" agents. However, it remains of first importance to guarantee some safety property under these conditions. For example, in the secret sharing protocol, we expect that a corrupted agent cannot reconstruct highly sensitive information before the other agents. Byzantine-Altruistic-Rational (BAR) models have been introduced in \cite{AACDMP05} in order to analyse the protocol correctness of systems under these conditions. In these systems, the agents are divided in three categories of players. They are said to be altruistic if they follow the rules of the protocol, even if a rational choice is to deviate from it. The agents are rational if they may deviate from the protocol in order to maximise their self-interest, captured in a utility function. Finally, Byzantine agents characterise the players who arbitrarily deviate from the protocol because of component failures, malicious intent, security compromise. The general architecture of BAR systems have been introduced in \cite{AACDMP05}. In the BAR framework, the correctness of the protocol with respect to some property is satisfied if the property remains true in spite of the presence of rational and Byzantine players. Such protocols are said BAR-tolerant. Some of them have been already proposed to implement cooperative services in peer-to-peer data streaming applications \cite{LCWNRAD06}.

BAR-tolerant protocols are thus important because, they guarantee that selfish agents actually play as expected by the protocol even if some agents are identified as broken or misconfigured. However, it remains challenging to prove that a protocol is BAR-tolerant. For that purpose, it is relevant to make use of Nash-equilibriums, a common game theory concept characterising strategies from which rational players should not deviate in order to maximise their utility. In \cite{MMSTACL08}, the authors presented a symbolic Model Checking algorithm that automatically verifies whether the proposed protocol is a \emph{Nash-equilibrium}. In \cite{Brenguier13}, PRALINE has been presented as a tool for computing Nash-equilibrium in non-probabilistic concurrent games played on graphs. By concurrent games, we mean games where the next state is defined at the beginning of a turn by the individual choices of all the players. However, these previous works are limited to non-\emph{probabilistic protocols}. Moreover, the game must be modelled with perfect information, meaning that the rational agents are assumed to have a perfect knowledge of the player utilities and of the global state of the game.

In our article, we focus on a particular class of probabilistic protocols, that is  protocols in which the agents take decisions accordingly to a specified probability distribution. The altruistic agents play thus with respect to this distribution, the rational agents play  with respect to the probability distribution that maximises their expected utility and the Byzantine agents play accordingly to any probability distribution. Note that we do not cover the case in which illegal decisions with respect to the BAR system may be taken by the rational and Byzantine players. Moreover, we consider probabilistic games in which the utilities are known by the rational agents, as well as the number of Byzantine players.

Related to our work, the tool PRISM-games \cite{CFKPS13} computes an optimal strategy for a coalition of rational players in a probabilistic game. This strategy can then be used to verify that the global strategy is a Nash-equilibrium. However, PRISM-games is limited to the analysis of turn-based games, that are games where agents select their moves in turns. Synchronisation between agents is thus not available directly in PRISM-games. Moreover, the protocols are with perfect information: the agents know the whole reward structure as well as the current global state of the game and the probability distributions of all the players. Finally, PRISM-games computes an optimal strategy up to some bounded number of iterations but it does not compute explicitly a potentially larger bound guaranteeing that a particular strategy is a Nash-equilibrium.

\subsection{Contributions and structure of the article}
In Section \ref{sec:spec}, we present the specification of parallel and synchronous probabilistic BAR systems. In Section \ref{sec:verification}, we propose a verification algorithm to verify Nash-equilibrium of the system. Finally, we apply the algorithm in two case studies, one with perfect information and one with imperfect information in Section \ref{sec:casestudy} and we present interesting results with regard of Byzantine agents. We summarise our results in Section \ref{sec:conclusion}. For the best of our knowledge, this is the first attempt of verifying Nash-equilibrium of probabilistic BAR systems. Nash-equilibrium is verified for imperfect information scenario as well as the perfect information scenario. This differs from previous works e.g.,~\cite{CFKPS13} and~\cite{MMSTACL08}, since we consider imperfect information games and since we verify Nash-equilibriums explicitly instead of approximate equilibriums \cite{CFKPS13}. Finally, our verification algorithm determines a bound for the number of iterations that guarantees the Nash-equilibrium property.    
