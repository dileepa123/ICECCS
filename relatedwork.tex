\section{Introduction}
Game Theory is the study of mathematical models of conflict and cooperation between multiple agents, considered as intelligent and rational decision-makers and is used in a large range of domains (economics, psychology, computer science...). In communication networks, the protocols are seen as games in which the involved entities of the network must cooperatively follow the protocol rules. However, it may be possible that a particular entity is corrupted or does not work correctly with respect to the protocol. Some device may have been designed to deviate from the protocol to maximise a self-interest function or may be simply misconfigured. In this article, we reason on Byzantine-Altruistic-Rational (BAR) models \cite{AACDMP05}. In these systems, the agents are divided in three categories of players. They are said to be altruistic if they follow the rules of the protocol, even if a rational choice is to deviate from it. The agents are rational if they may deviate from the protocol in order to maximise their self-interest, captured in a utility function. Finally, Byzantine agents characterise the players who arbitrarily deviate from the protocol because of component failures, malicious intent, security compromise. A general architecture of BAR systems have been introduced in \cite{AACDMP05}. In the BAR framework, the correctness of the protocol with respect to some property is satisfied if the property remains true in spite of the presence of rational and Byzantine players. Such protocols are said BAR-tolerant and some of them have already been proposed to implement cooperative services in peer-to-peer data streaming applications \cite{LCWNRAD06}.

BAR-tolerant protocols are thus important because they guarantee that selfish agents actually play as expected bythe protocol even if some agents are identified as broken or misconfigured. However, it remains challenging to prove that a protocol is BAR-tolerant. For that purpose, it is relevant to make use of Nash equilibriums, a common game theory concept characterising strategies from which rational players should not deviate in order to maximise their utility. in \cite{MMSTACL08}, the authors presented a symbolic Model Checking algorithm that automatically verifies whether the proposed protocol is a \emph{Nash-equilibrium}. In \cite{Brenguier13}, PRALINE has been presented as a tool for computing Nash equilibrium in non-probabilistic concurrent games played on graphs. By concurrent games, we mean games where the next state is defined at the beginning of a turn by the individual choices of all the players. However, these previous works are limited to non-\emph{probabilistic protocols}. Moreover, the game must be modelled with perfect information, meaning that the rational agents are assumed to have a perfect knowledge of the player utilities and of the global state of the game.

In our article, we focus on a a particular class of probabilistic protocols, that are protocols in which the agents take decisions accordingly to a specified probability distribution. The altruistic agents play thus with respect to this distribution, the rational agents play  with respect to the probability distribution that maximises their expected utility and the Byzantine agents play accordingly to any probability distribution. Note that we do not cover the case in which illegal decisions with respect to the protocol may be taken by the rational and Byzantine players. Moreover, we consider probabilistic games in which the utilities are known by the rational agents, as well as the number of Byzantine players.

Related to our work, the tool PRISM-games \cite{CFKPS13} computes an optimal strategy for one rational player in a probabilistic game with perfect information. This strategy can then be used to verify that the global strategy is a Nash-equilibrium. However, PRISM-games is limited to the analysis of turn-based games, that are games where agents select their moves in turns. Synchronisation is thus not available directly in PRISM-games.