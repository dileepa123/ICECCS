
\renewcommand{\thesection}{\Alph{section}}%

\section{Proofs}\label{apdx}

\setcounter{theorem}{0}
\setcounter{proposition}{0}
\setcounter{lemma}{0}
%\appendix
%\section{Proofs}
%\label{apdx}

In general, given a set of Byzantine players $Z \subset [n]$, a global state $s$, 
%\in S_{|g}$ where $g=\langle g^1, \ldots, g^n\rangle$ with $g^i={\sf b}$ if $i\in Z$ and $g^i={\sf a}$ or $g^i={\sf r}$ if $i\notin Z$, 
the pay-off of $i$ playing the game for $k$ steps should be as follows.
\[
\begin{array}{l}
v_i^{'k}(Z,s)=\\
\hspace{1.2cm} \left\{
\begin{array}{l}
{\bf \text{if } k > 0 \wedge i \notin Z}  \\
max_{\pi'^{a}_{i} \in \Pi_{t=0}^{k-1} A_{i}}
(min_{\pi'^{a}_{Z} \in \Pi_{t=0}^{k-1} A_{Z}}
\{E_{\pi'^{a}_{[n]-Z-\{i\}} \in \Pi_{t=0}^{k-1} A_{[n]-Z-\{i\}}}\\ \hfill
(\Sigma_{t=0}^{k-1} \beta_i^{t}H_{i}(\pi'^{s}(t),\pi'^{a}(t))) |\\ BT(Z\cup\{i\},\pi'^{s}(t),\pi'^{a}(t),\pi'^{s}(t+1)) \wedge \pi'^{s}(0)=s \wedge |\pi'|=k\})\\
{\bf \text{if } k > 0 \wedge i \in Z} \\
max_{\pi'^{a}_{i} \in \Pi_{t=0}^{k-1} A_{i}}
min_{\pi'^{a}_{[Z]-\{i\}} \in \Pi_{t=0}^{k-1} A_{[Z]-\{i\}}}
\{\{E_{\pi'^{a}_{[n]-Z} \in \Pi_{t=0}^{k-1} A_{[n]-Z}}\\ \hfill(\Sigma_{t=0}^{k-1} \beta_i^{t}H_{i}(\pi'^{s}(t),\pi'^{a}(t))) |\\ BT(Z,\pi'^{s}(t),\pi'^{a}(t),\pi'^{s}(t+1)) \wedge \pi'^{s}(0)=s \wedge |\pi'|=k\}\}\\
{\bf \text{if } k = 0}  \\0
\end{array}
\right.
\end{array}
\]
where $\pi'$ is a path of length $k$.

In the case of $k > 0 \wedge i \notin Z$, $\pi'^{s}(t)$ is the state at position $t$ in the path $\pi'$
and $\pi'^{a}(t)$ is the action at position $t$ in the path $\pi'$.
Hence, $H_{i}(\pi'^{s}(t),\pi'^{a}(t))$ defines the pay-off of $i$ on the transition at position $t$.
The pay-offs of length $k$ is the sum of each position, with every pay-off weighted with discount factor $\beta_i^t$, i.e., $\Sigma_{t=0}^{k-1} \beta_i^{t}H_{i}(\pi'^{s}(t),\pi'^{a}(t))$. Note that given a choice, we have a tree, containing a set of paths.
For each path in the tree, calculation of the pay-off of the path is as above. 
The expected pay-off of the tree can be calculated using these pay-offs, by considering the probabilities in each path.
This process is denoted by $E_{\pi'^{a}_{[n]-Z-\{i\}} \in \Pi_{t=0}^{k-1} A_{[n]-Z-\{i\}}}$.
The possible trees are grouped by the rational players' choices of non-deterministic actions, i.e., for one choice, there are a set of trees due to that there
may be different choices for Byzantine players. For each set/group of trees,
we choose the tree that minimised the expected pay-off, denoted by $min_{\pi'^{a}_{Z} \in \Pi_{t=0}^{k-1} A_{Z}}$,
because we assume that the Byzantine players try to minimise $i$'s pay-offs.
Now in each group (i.e., for a choice of the rational player), there is only the minimised tree,
and each group has exactly one choice of non-deterministic actions of $i$.
We choose the one which gives the maximum expected pay-off, denoted by $max_{\pi'^{a}_{i} \in \Pi_{t=0}^{k-1} A_{i}}$, meaning that the rational player $i$
always makes the choice that gives the maximised pay-off. 
In addition, we ensure that each transition in each path is valid ($BT(Z\cup\{i\},\pi'^{s}(t),\pi'^{a}(t),\pi'^{s}(t+1))$),
the initial state of each path is $s$ and the length of each path is $k$. Hence, 
in summary, the formula captures the intuitive calculate of $i$'s pay-off in length $k$ starting from $s$ w.r.t. $Z\cup\{i\}$.

If $k > 0 \wedge i \in Z$, since $i\in Z$, the set of altruistic and Byzantine players
differ from the set of altruistic and Byzantine players in the case of $i\notin Z$, that is,
in the case of $i\notin Z$, $i$ can be altruistic or rational, whereas in the case of $i\in Z$,
$i$ can be Byzantine or rational. Hence, the grouping of trees due to the rational players' choices
is different in these two cases. Therefore, in the case of $i\in Z$, the process of calculating the
expected pay-off of trees is denoted differently as $E_{\pi'^{a}_{[n]-Z} \in \Pi_{t=0}^{k-1} A_{[n]-Z}}$.
The second difference is in the $BT$ functions. Since $i\in Z$, we do not need to additionally add $i$ to $Z$ to
capture the rational behaviour of $i$. 
If $k=0$, we initialise the pay-off as $0$.

Similarly, we define the correct pay-off for $u_i^{'k}(Z,s)$ as follows:
%
%\[
%\begin{array}{l}
%v_i^{'k}(Z,s)=\\
%\hspace{1.2cm} \left\{
%\begin{array}{l}
%{\bf \text{if } k > 0 \wedge i \notin Z}  \\
%max_{\pi'^{a}_{i} \in \Pi_{t=0}^{k-1} A_{i}}
%(min_{\pi'^{a}_{Z} \in \Pi_{t=0}^{k-1} A_{Z}}
%\{E_{\pi'^{a}_{[n]-Z-\{i\}} \in \Pi_{t=0}^{k-1} A_{[n]-Z-\{i\}}}\\ \hfill
%(\Sigma_{t=0}^{k-1} \beta_i^{t}H_{i}(\pi'^{s}(t),\pi'^{a}(t))) |\\ BT(Z\cup\{i\},\pi'^{s}(t),\pi'^{a}(t),\pi'^{s}(t+1)) \wedge \pi'^{s}(0)=s \wedge |\pi'|=k\})\\
%{\bf \text{if } k > 0 \wedge i \in Z} \\
%max_{\pi'^{a}_{i} \in \Pi_{t=0}^{k-1} A_{i}}
%min_{\pi'^{a}_{[Z]-\{i\}} \in \Pi_{t=0}^{k-1} A_{[Z]-\{i\}}}
%\{\{E_{\pi'^{a}_{[n]-Z} \in \Pi_{t=0}^{k-1} A_{[n]-Z}}\\ \hfill(\Sigma_{t=0}^{k-1} \beta_i^{t}H_{i}(\pi'^{s}(t),\pi'^{a}(t))) |\\ BT(Z,\pi'^{s}(t),\pi'^{a}(t),\pi'^{s}(t+1)) \wedge \pi'^{s}(0)=s \wedge |\pi'|=k\}\}\\
%{\bf \text{if } k = 0}  \\0
%\end{array}
%\right.
%\end{array}
%\]
%
\[
\begin{array}{l}
u_{i}^{'k}(Z,s)=
\left\{
\begin{array}{l}
{\bf \text{if } k > 0 \wedge i \notin Z}  \\
E_{a_i \in A_i} \big(
min _{a_{Z} \in A_{Z}}\{
\\E_{a_{[n]-Z-\{i\}} \in A_{a_{[n]-Z-\{i\}}}} (\Sigma_{t=0}^{k-1} \beta_i^{t}H_{i}(\pi'^{s}(t),\pi'^{a}(t))) |\\ BT(Z,\pi'^{s}(t),\pi'^{a}(t),\pi'^{s}(t+1)) \wedge \pi'^{s}(0)=s \wedge |\pi'|=k\})\\
{\bf \text{if } k > 0 \wedge i \in Z} \\
min _{a_i \in A_i} \{
min_{a_{Z-\{i\}} \in A_{Z-\{i\}}}\{
\\E_{a_{[n]-Z} \in A_{[n]-Z}} (\Sigma_{t=0}^{k-1} \beta_i^{t}H_{i}(\pi'^{s}(t),\pi'^{a}(t))) |\\ BT(Z,\pi'^{s}(t),\pi'^{a}(t),\pi'^{s}(t+1)) \wedge \pi'^{s}(0)=s \wedge |\pi'|=k\})\\
\end{array}
\right.
\end{array}
\]

\subsection{Correctness of the dynamic programming definition}
%Since we have probabilistic strategies we do not always consider paths. We consider a set of paths to which probabilities are assigned. We optimize the expected value of these paths. Let $\pi'^a$ be a path of length $k$, $Z \subset [n]$  is the Byzantine player set and $s \in S_{z}$ a global state. The original value function for a horizon $t$ is given by:
%%$v_i^{k}(Z,s)=\max_{\sigma \in Strat_{k}(s,Z,i)}$\\
%%$\{\min_{\pi \in Path(s,Z,i,\sigma)}E_{\Pi_{t=0}^{k-1}A_{[n]-Z}(\pi^(s)(t))}(\Sigma_{t=0}^{k-1} \beta_i^{t}h_{i}(\pi^{s}(t),\pi^{s}(a)))\}$ \newline
%
%%$v_i^{k}(Z,s)= max_{\sigma \in Strat_{k}(s,Z,i)}\{min_{}  \} $
%%$v_i^{k}(Z,s)=E_{\pi'^{a} \in \Pi_{t=0}^{k-1} A_{[n]-Z}}(\Sigma_{t=0}^{k-1} \beta_i^{t}h_{i}(\pi'^{s}(t),\pi'^{a}(t)))$
%
\[
\begin{array}{l}
v_i^{'k}(Z,s)=\\
\hspace{1.2cm} \left\{
\begin{array}{l}
{\bf \text{if } k > 0 \wedge i \in Z} \\
max_{\pi'^{a}_{i} \in \Pi_{t=0}^{k-1} A_{i}}
min_{\pi'^{a}_{[Z]-\{i\}} \in \Pi_{t=0}^{k-1} A_{[Z]-\{i\}}}
\{\{E_{\pi'^{a}_{[n]-Z} \in \Pi_{t=0}^{k-1} A_{[n]-Z}}\\ \hfill(\Sigma_{t=0}^{k-1} \beta_i^{t}h_{i}(\pi'^{s}(t),\pi'^{a}(t))) |\\ BT(Z,\pi'^{s}(t),\pi'^{a}(t),\pi'^{s}(t+1)) \wedge \pi'^{s}(0)=s \wedge |\pi'|=k\}\}\\
{\bf \text{if } k > 0 \wedge i \notin Z}  \\
E_{\pi'^{a}_{i} \in \Pi_{t=0}^{k-1} A_{i}}
(min_{\pi'^{a}_{Z} \in \Pi_{t=0}^{k-1} A_{Z}}
\{E_{\pi'^{a}_{[n]-Z-\{i\}} \in \Pi_{t=0}^{k-1} A_{[n]-Z-\{i\}}}\\ \hfill
(\Sigma_{t=0}^{k-1} \beta_i^{t}h_{i}(\pi'^{s}(t),\pi'^{a}(t))) |\\ BT(Z,\pi'^{s}(t),\pi'^{a}(t),\pi'^{s}(t+1)) \wedge \pi'^{s}(0)=s \wedge |\pi'|=k\})\\
{\bf \text{if } k = 0}  \\0
\end{array}
\right.
\end{array}
\]

%We can use induction to prove that the dynamic programming definition of $v_i^{k}(Z,s)$ also computes the same function. Induction is done on the path length $t$ \cite{MMS08}.
For recall,

\begin{theorem}
	$v_i^{'k}(Z,s)=v_i^{k}(Z,s)$ and $u_i^{'k}(Z,s)=u_i^{k}(Z,s)$, $ \forall k \ge 0 $.
\end{theorem}

\begin{proof}
	For $k=0$, the result is trivial because value of empty path is defined to be $0$. So the maximin path value is 0.
	Substitute the original function value $v_i^{'k}(Z,s)$ to compute $v_i^{k+1}(Z,s)$ in the dynamic programming definition. 
	Let $i \in Z$, 
	%let the new path set be $\{\pi'\}$ and the 
	any path of length $k$  th iteration be $\pi$ and let $\{\pi\}$ be any infinite length probabilistic path starting from state $s$.
	%	$k$ corresponds to a length of $k+1$. $(k \geq 0)$
	%	$E_{A_{[n]-Z}(s)}(h(s,<a_i,a_{-i}>)+\beta_iv_i^{k}(Z,s'))=$\\
	%	$E_{A_{[n]-Z}(s)}(h(s,<a_i,a_{-i}>)+\beta_iE_{\pi^{a} \in \Pi_{t=0}^k A_{[n]-Z}}(\Sigma_{t=0}^{k} \beta_i^{t}h_{i}(\pi^{s}(t),\pi^{a}(t))))$\\
	%	$E_{A_{[n]-Z}(s)}(h(s,<a_i,a_{-i}>)) +$
	%	$ E_{A_{[n]-Z}(s)}(E_{\pi'^{a} \in \Pi_{t=1}^{k+1}A_{[n]-Z}}(\Sigma_{t=1}^{k+1} \beta_i^{t}h_{i}(\pi'^{s}(t),\pi'^{a}(t))))$\\
	%	$E_{\pi'^{a} \in \Pi_{t=0}^{k+1}A_{[n]-Z}}(h(s,<a_i,a_{-i}>)) +$
	%	$ E_{\pi'^{a} \in \Pi_{t=0}^{k+1} A_{[n]-Z}}(\Sigma_{t=1}^{k+1} \beta_i^{t}h_{i}(\pi'^{s}(t),\pi'^{a}(t)))$\\
	%	$E_{\pi'^{a} \in \Pi_{t=0}^{k+1} A_{[n]-Z}}(\Sigma_{t=0}^{k+1} \beta_i^{t}h_{i}(\pi'^{s}(t),\pi'^{a}(t)))$\\
	
%	\begin{equation}
%	\begin{split}
%	E_{a_{[n]-Z} \in A_{[n]-Z}}(h(s,\langle a_i,a_{-i} \rangle)+\beta_iv_i^{k}(Z,s')|BT(Z,s,\langle a_i,a_{-i} \rangle,s'))
%	\\
%	=E_{a_{[n]-Z} \in A_{[n]-Z}}(h(s,\langle a_i,a_{-i} \rangle)+\beta_iE_{\pi^{a} \in \Pi_{t=0}^{k-1} A_{[n]-Z}}(\Sigma_{t=0}^{k-1} \beta_i^{t}h_{i}(\pi^{s}(t),\pi^{a}(t))|BT(Z,s,\langle a_i,a_{-i} \rangle,s')))\\
%	=E_{a_{[n]-Z} \in A_{[n]-Z}}(h(s,\langle a_i,a_{-i} \rangle)) +
%	E_{A_{[n]-Z}}(E_{\pi'^{a} \in \Pi_{t=1}^{k}A_{[n]-Z}}(\Sigma_{t=1}^{k} \beta_i^{t}h_{i}(\pi'^{s}(t),\pi'^{a}(t))|BT(Z,s,\langle a_i,a_{-i} \rangle,s')))\\
%	=E_{\pi'^{a} \in \Pi_{t=0}^{k}A_{[n]-Z}}(h(s,\langle a_i,a_{-i} \rangle)) +
%	E_{\pi'^{a} \in \Pi_{t=0}^{k} A_{[n]-Z}}(\Sigma_{t=1}^{k} \beta_i^{t}h_{i}(\pi'^{s}(t),\pi'^{a}(t))|BT(Z,s,\langle a_i,a_{-i} \rangle,s'))\\
%	=E_{\pi'^{a} \in \Pi_{t=0}^{k} A_{[n]-Z}}(\Sigma_{t=0}^{k} \beta_i^{t}h_{i}(\pi'^{s}(t),\pi'^{a}(t))|BT(Z,s,\langle a_i,a_{-i} \rangle,s'))\\
%	\end{split}
%	\end{equation}

	\begin{equation}
	\begin{split}
	&E_{a_{[n]-Z} \in A_{[n]-Z}}(h(s,\langle a_i,a_{-i} \rangle) +\beta_iv_i^{k}(Z,s')|BT(Z,s,\langle a_i,a_{-i} \rangle,s')) \\
& = E_{a_{[n]-Z} \in A_{[n]-Z}}(h(s,\langle a_i,a_{-i} \rangle)+\beta_iE_{\pi^{a} \in \Pi_{t=0}^{k-1} A_{[n]-Z}}(\Sigma_{t=0}^{k-1} \beta_i^{t}h_{i}(\pi^{s}(t),\pi^{a}(t))|\\ &\hspace{2cm} BT(Z,s,\langle a_i,a_{-i} \rangle,s')))\\
& =E_{a_{[n]-Z} \in A_{[n]-Z}}(h(s,\langle a_i,a_{-i} \rangle)) +
	E_{A_{[n]-Z}}(E_{\pi'^{a} \in \Pi_{t=1}^{k}A_{[n]-Z}}(\\ &\hspace{2cm} \Sigma_{t=1}^{k} \beta_i^{t}h_{i}(\pi'^{s}(t),\pi'^{a}(t))|BT(Z,s,\langle a_i,a_{-i} \rangle,s')))\\
& =E_{\pi'^{a} \in \Pi_{t=0}^{k}A_{[n]-Z}}(h(s,\langle a_i,a_{-i} \rangle)) +
	E_{\pi'^{a} \in \Pi_{t=0}^{k} A_{[n]-Z}}(\Sigma_{t=1}^{k} \beta_i^{t}h_{i}(\pi'^{s}(t),\pi'^{a}(t))|\\ &\hspace{2cm} BT(Z,s,\langle a_i,a_{-i} \rangle,s'))\\
& =E_{\pi'^{a} \in \Pi_{t=0}^{k} A_{[n]-Z}}(\Sigma_{t=0}^{k} \beta_i^{t}h_{i}(\pi'^{s}(t),\pi'^{a}(t))|BT(Z,s,\langle a_i,a_{-i} \rangle,s'))
	\end{split}
	\end{equation}	
	
The maximin value would be:
	\begin{multline}
	v_i'k(Z,s) = max_{\pi'^{a}_{i} \in \Pi_{t=0}^{k} A_{i}}
	min_{\pi'^{a}_{[Z]-\{i\}} \in \Pi_{t=0}^{k} A_{[Z]-\{i\}}}\\
	E_{\pi'^{a} \in \Pi_{t=0}^{k} A_{[n]-Z}}(\Sigma_{t=0}^{k} \beta_i^{t}h_{i}(\pi'^{s}(t),\pi'^{a}(t))|BT(Z,s,\langle a_i,a_{-i} \rangle,s'))
	\end{multline}
	
	When $\{\pi\}$ is the path set corresponding to optimal expected value for length $k$,   $\{\pi'\}$ would be the path set $s\langle a_i,a_{-i}\rangle \pi \in \Pi_{t=0}^{k} A_{[n]-Z}$. Suppose we choose a different path set from $s'$ other than $v_i^k(s',Z)$ if does not correspond to the maximum expected value over player $i$ among the guaranteed values, we can miss a max value path set in general. The proof is similar for $u_i^k(Z,s)$.  %\qed
\end{proof}

\subsection{Correctness of Arbitrary Precision Nash Equilibria Verification\cite{MMS08}}
 
\paragraph{Proposition \ref{prop:limkvu}} Though this result is not directly used in the proof of Theorem \ref{thm:correctness}, we take advantage of this subsection to explain how one can prove it. Indeed, Proposition \ref{prop:limkvu} relies on the same Lemmas than the ones used for the proof of Theorem \ref{thm:correctness}.
For recall,
\begin{proposition}\label{prop:limkvu}
	\begin{enumerate}
		\item $\lim_{k\to\infty} v_i^{k}(Z,s)=v_i(Z,s)$
		\item $\lim_{k\to\infty} u_i^{k}(Z,s)=u_i(Z,s)$
	\end{enumerate}
\end{proposition}
\begin{lemma}\label{lem:limvkpi}
	\begin{enumerate}
		\item $\forall k\in \mathbb{N}, |v_i(\{\pi\})-v_i(\{\pi|_k\})|$ $\le$ $e_i(k)$.	
		\item $\lim_{k\to\infty}  v_i(\{\pi|_k\})=v_i(\{\pi\})$.
	\end{enumerate}
\end{lemma}
\begin{lemma}\label{lem:vdiff}
	Let $\{ \pi \}$ be a probabilistic path set s.t. $v_i(\{\pi\})$ is minimum in ${\sf Path}(s,Z,i,\sigma)$, let $\{\{\bar\pi_k \} \}_{k \in \mathbb{N}}$ be a sequence of finite path, s.t. $\forall k$, $v_i(\{\bar\pi_k   \})$ is minimum in ${\sf Path}_k(s,Z,i,\sigma)$, then $v_i(\{\pi|_k\})-v_i(\{\bar\pi_k\}) \le 2e_i(k)$.
\end{lemma}
\begin{lemma}\label{lem:limvkz}
	$\forall s\in S$ and strategy $\sigma$ we have:\\
	$\lim_{k\to\infty} v_i^{k}(Z,s,\sigma|_k) = v_i(Z,s,\sigma) $.
\end{lemma}
\begin{lemma}\label{lem:vkvdiff}
	$\forall  s\in S$ and $\forall k \in \mathbb{N}$, we have:\\
	$|v_i^{k}(Z,s)-v_i(Z,s)| < E_i(k)$.
\end{lemma}
\begin{lemma}\label{lem:ukudiff}
	$\forall s\in S$ and $\forall k \in \mathbb{N}$, we have: \\
	$|u_i^{k}(Z,s) - u_i(Z,s)| < E_i(k)$.
\end{lemma}
As written in the paper, apart Lemma \ref{lem:limvkpi}, Lemmas \ref{lem:vdiff} to \ref{lem:ukudiff}, as well as the proposition, are very similar from the results established in \cite{MariPhD}. For that reason, we do not reproduce their proof here and we refer to the text and \cite{MariPhD} for more details. Proof of Lemma \ref{lem:limvkpi} is given below.

We now present the proof of Theorem \ref{thm:correctness}. For recall,
\begin{theorem}%\label{thm:correctness}
	Let $\mathcal{G}=(S, s_0, A, T, P, H, \beta)$ be an $n$ player game, $ \epsilon > 0$ and $\delta > 0$ and $Z \subset [n]$ be the set of Byzantine players, for each player $i \in [n]$ let 
	\begin{enumerate}
		\item $M_i$ = $max \{|h_i(s,a)| | s \in S$ and $a \in A\}$
		\item $E_i(k) = 5\beta_i^{k} \frac{M_i}{1-\beta_i}$
		\item $\Delta_i(k)=max\{v_i^{k}(Z \cup \{i\},{\sf f}(s))- u_i^{k}(Z,s)|s\in O \}$
		\item $\epsilon_1(i,k) = \Delta_i(k) - 2E_i(k)$
		\item $\epsilon_2(i,k) = \Delta_i(k) + 2E_i(k)$,
	\end{enumerate}
	and let $k_i$ be the minimum numbers of steps such that $4E_i(k_i) < \delta$,
	\begin{enumerate}
		\item \text{if } $\forall i \in [n]$, $\epsilon \ge \epsilon_2(i,k_i) > 0$ then $\mathcal{M}_{|gz}$ is $\epsilon$-Nash-equilibrium,
		\item \text{if } $\exists i \in [n]$, $0 < \epsilon \le \epsilon_1(i,k_i)$ then $\mathcal{M}_{|gz}$ is not $\epsilon$-Nash-equilibrium,
		\item \text{if } $\forall i \in [n]$, $\epsilon_1(i,k_i) < \epsilon$ and $\exists j \in [n]$ s.t. $\epsilon < \epsilon_2(j,k_j)$ then $\mathcal{M}_{|gz}$ is $(\epsilon + \delta)$-Nash-equilibrium.
	\end{enumerate}
\end{theorem}

\begin{proof}
In order to prove the convergence of the value function, we want to bound the value difference by a more convenient function. For that purpose, we define $e_i(k)=\beta_i^{k}\frac{M_i}{1-\beta_i}$ and prove Lemma \ref{lem:limvkpi}.
\setcounter{lemma}{0}
\begin{lemma}\label{lem:limvkpi}
	\begin{enumerate}
		\item $\forall k\in \mathbb{N}, |v_i(\{\pi\})-v_i(\{\pi|_k\})|$ $\le$ $e_i(k)$.	
		\item $\lim_{k\to\infty}  v_i(\{\pi|_k\})=v_i(\{\pi\})$
	\end{enumerate}
\end{lemma}

\begin{proof}
	%	Let $\pi''^{a}=<b,\pi'^{a}>$ , $b \in A$\newline
	\begin{enumerate}
		\item 	$|v_i(\{\pi|_T\})-v_i(\{\pi|_k\})| =$\\
		$~~~~~| E_{\pi^{'a} \in \Pi_{t=0}^{k-1} A_{[n]-Z}}(\Sigma_{t=0}^{k-1} \beta_i^{t}h_{i}(\pi^{s}(t),\pi^{a}(t))) -$\\
		$~~~~~~ E_{\pi''^{a} \in \Pi_{t=0}^{k-1} A_{[n]-Z} \Pi_{t=k}^{T} A_{[n]-Z}}(\Sigma_{t=0}^{T} \beta_i^{t}h_{i}(\pi''^{s}(t),\pi''^{a}(t)))|$\\
		$=| E_{\pi''^{a} \in \Pi_{t=0}^{k-1} A_{[n]-Z} \Pi_{t=k}^{T} A_{[n]-Z}}(\Sigma_{t=0}^{k-1} \beta_i^{t}h_{i}(\pi^{s}(t),\pi^{a}(t)))-$\\
		$~~~~~~ E_{\pi''^{a} \in \Pi_{t=0}^{k-1} A_{[n]-Z} \Pi_{t=k}^{T} A_{[n]-Z}}(\Sigma_{t=0}^{T} \beta_i^{t}h_{i}(\pi''^{s}(t),\pi''^{a}(t)))|$\\
		$ =|E_{\pi''^{a} \in \Pi_{t=0}^{k-1} A_{[n]-Z} \Pi_{t=k}^{T} A_{[n]-Z}}(\Sigma_{t=k}^{T} \beta_i^{t}h_{i}(\pi''^{s}(t),\pi''^{a}(t)))|$\\
		$\leq |\Sigma_{t=k}^{T} E_{\pi''^{a} \in \Pi_{t=0}^{k-1} A_{[n]-Z} \Pi_{t=k}^{T} A_{[n]-Z}}( \beta_i^{t}h_{i}(\pi''^{s}(t),\pi''^{a}(t)))| $ (Linearity of Expectation)\\
		$\leq \Sigma_{t=k}^{T}| E_{\pi''^{a} \in \Pi_{t=0}^{k-1} A_{[n]-Z} \Pi_{t=k}^{T} A_{[n]-Z}}( \beta_i^{t}h_{i}(\pi''^{s}(t),\pi''^{a}(t)))| $ (Triangle inequality)\\
		$\le \beta_i^{k}\frac{M_i}{1-\beta_i}$ (By the choice of $M_i$ and $\lim_{T\to\infty}$)\\
		$\leq e_i(k)$
		\item $\lim_{k\to\infty} e_i(k)=0$,\\
		$\lim_{k\to\infty} |v_i(\{\pi\})-v_i(\{\pi|_k\})| = 0$ (by comparison test)\\
		$\lim_{k\to\infty} v_i(\{\pi|_k\})=v_i(\{\pi\})$.
	\end{enumerate}
%	\qed
\end{proof}


Now we have all the intermediate results to prove Theorem \ref{thm:correctness}. By Lemma \ref{lem : conv_vi7} and \ref{lem : conv_ui8}, we have:  $\forall s \in S,\;|v_i^{k}(Z,s)-v_i(Z,s)| < E_i(k)$ and $|u_i^{k}(Z,s) - u_i(Z,s)| < E_i(k)$. This implies:\\
$v_i(Z \cup \{i\},s) \le v_i^{k}(Z \cup \{i\},s) + E_i(k)$  by Lemma $ \ref{lem : opt_finite_path5}$\\
$v_i(Z \cup \{i\},s) \ge v_i^{k}(Z \cup \{i\},s) - E_i(k)$  by Lemma $ \ref{lem : opt_finite_strat6}$\\
$u_i(Z ,s) \le u_i^{k}(Z ,s) + E_i(k)$  by Lemma $ \ref{lem : conv_vi7}$\\
$u_i(Z ,s) \ge u_i^{k}(Z ,s) - E_i(k)$  by Lemma $ \ref{lem : conv_ui8}$\\

Now, we can prove the three following statements:
\begin{enumerate}
	\item Using Lemma \ref{lem : opt_finite_path5} and Lemma \ref{lem : conv_ui8},\\
	$v_i(Z \cup \{i\},s) - u_i(Z ,s) \le v_i^{k}(Z \cup \{i\},s) + E_i(k) - (u_i^{k}(Z ,s) - E_i(k) )$\\
	$=v_i^{k}(Z \cup \{i\},s) - u_i^{k}(Z ,s) + 2E_i(k)  $\\
	$\le \Delta_i(k) + 2E_i(k) $\\
	if $\epsilon \ge \epsilon_2(i,k)$ then $\Delta_i(k) \le \epsilon - 2E_i(k)$\\
	So, $\forall s \in I$, \\
	$v_i(Z \cup \{i\},s) - u_i(Z ,s) \le \epsilon$\\
	$M$ is $\epsilon-Nash$.
	\item Similarly, \ref{lem : opt_finite_strat6} and \ref{lem : conv_vi7} can be used to prove\\
	$v_i(Z \cup \{i\},s) - u_i(Z ,s) \ge v_i^{k}(Z \cup \{i\},s) - E_i(k) - (u_i^{k}(Z ,s) + E_i(k) )$\\
	$=v_i^{k}(Z \cup \{i\},s) - u_i^{k}(Z ,s) - 2E_i(k)  $\\
	if $\epsilon \le \epsilon_1(i,k)$ then $\Delta_i(k) \ge \epsilon + 2E_i(k)$\\
	This implies $\exists Z \in P([n]-\{i\})$ and $s \in I$ s.t.\\
	$v_i(Z \cup \{i\},s) - u_i(Z ,s) \ge \epsilon$\\
	$M$ is not $\epsilon-Nash$.
	\item if $\forall i$, $\epsilon_1(i,k_i) < \epsilon$ and for some $j, \epsilon < \epsilon_2(j,k_j)$, \\
	it is not possible to decide whether $M$ is $\epsilon-Nash$. But, since $\epsilon_2(j,k_j)-\epsilon_1(i,k_i)= 4E_i(k_i)$ and $4E_i(k_i) < \delta$, we have,\\
	$\forall i \in [n]$, $\epsilon + \delta > \epsilon_2(i,k_i)$. According to the first statement, we have $(\epsilon + \delta)-Nash$.
\end{enumerate}
%\qed
\end{proof}


